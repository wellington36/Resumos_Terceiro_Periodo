\documentclass[12pt]{article}

\usepackage{setspace}
\usepackage{amssymb}
\usepackage{amsmath}
\usepackage{amsfonts}
\usepackage{wasysym}
\usepackage{graphicx}
\usepackage[pdftex,bookmarks=true,bookmarksopen=false,bookmarksnumbered=true,colorlinks=true,linkcolor=black]{hyperref}
\usepackage[utf8]{inputenc}
\usepackage{float}
\usepackage{pdfpages}
\usepackage{enumerate}



\usepackage[brazil]{babel}

\pagestyle{plain}

\newtheorem{theorem}{Teorema}[section]
\newtheorem{corollary}{Corolário}[theorem]
\newtheorem{lemma}[theorem]{Lema}
\newtheorem{definition}{Definição}

\begin{document}

\begin{titlepage}
\begin{center}
\textbf{\LARGE Fundação Getulio Vargas}\\ 
\textbf{\LARGE Escola de Matemática Aplicada}

\par
\vspace{170pt}
\textbf{\Large Wellington José}\\
\vspace{32pt}
\textbf{\Large Resumo de Teoria da Probabilidade}\\
\end{center}

\par
\vfill
\begin{center}
{{\normalsize Rio de Janeiro}\\
{\normalsize \the\year}}
\end{center}
\end{titlepage}

\section{Conceitos Básicos}
\subsection*{1.2   Modelos de Probabilidade}
\begin{definition}
Dois eventos A e B são chamados de \textbf{mutuamente excludentes} se não podem ocorrer simultaneamente, isto é, se $A \cap B = \emptyset$.
\end{definition}

\begin{definition}
Uma \textbf{probabilidade} é uma função que associa a cada evento \textbf{A} um número \textbf{P(A)} de forma que:
\begin{enumerate}
    \item Para todo evento A, $0 \leqslant Pr(A) \leqslant 1$;
    \item $P(S) = 1$;
    \item Se A e B são eventos mutuamente excludentes então
    $$P(A \cup B) = P(A) + P(B)$$
\end{enumerate}
\end{definition}

\begin{corollary}[Lei do Complemento]
$$P(\overline{A}) = 1 - P(A)$$

Em outras palavras, a probabilidade de um evento ocorrer mais a probabilidade de ele não ocorrer dá $100\%$
\end{corollary}

\begin{corollary}
$P(\emptyset) = 0$, isto é se um evento é impossível, sua probabilidade deve ser 0.
\end{corollary}

\begin{corollary}[Lei da Adição]
$$P(A \cup B) = P(A) + P(B) - P(A \cap B)$$
\end{corollary}

\subsection*{Probabilidade Condicional}
\begin{definition}
Sejam A e B dois eventos com $P(A) \neq 0$. A probabilidade condicional de B dado A é

$$P(B|A) = \dfrac{P(A \cap B)}{P(A)}$$
\end{definition}

\begin{corollary}[Lei da Multiplicação]

$$P(A \cap B) = P(B|A) \cdot P(A) = P(B|A) \cdot P(B)$$

\end{corollary}

\subsubsection*{Probabilidade Total e Teorema de Bayes}
\begin{corollary}[Lei da Probabilidade Total]
Suponha que $B_1, B_2, \dots, B_n$ formam uma partição de S. Então

% $$P(A) = P(A \cap B_1) + \dotsb + P(A \cap B_n)$$

\end{corollary}

\begin{corollary}[Teorema de Bayes]
Suponha que $B_1, B_2, \cdots, B_n$ formam uma partição de S. Então

$$P(B_1|A) = \dfrac{P(A|B_1) P(B_1)}{P(A|B_1) P(B_1) + P(A|B_2) P(B_2) + \cdots + P(A|B_n) P(B_n)}$$

\end{corollary}

\subsubsection*{Independência}
\begin{definition}
Dois eventos (não impossíveis) A e B são ditos \textbf{independentes} se o conhecimento de um deles não afeta a probabilidade do outro ocorrer, isto é, se

$$P(B|A) = P(B)$$
\end{definition}

\section{Variáveis Aleatórias Discretas}
\subsection*{2.1 Função de Probabilidade e Função de Probabilidade Conjunta}
\begin{definition}
Se X é uma variáveis aleatórias discretas, definimos a \textbf{função de probabilidade de X} por

$$p_{X} (x) = P(X = x)$$
\end{definition}

\begin{definition}
Se X e Y são variáveis aleatórias discretas, definimos a \textbf{função de probabilidade conjunta de X e Y} por

$$p_{X, Y} (x, y) = P(X = x \text{ e } Y = y)$$
\end{definition}

\subsubsection*{Independência de Variáveis Aleatórias Discretas}
\begin{definition}
Dizemos que X e Y são variáveis \textbf{independentes} exatamente quando

$$P(X = i \text{ e } Y = j) = P(X = i) \cdot P(Y = j)$$
\end{definition}

\begin{corollary}
X e Y são independentes se, e somente se, a distribuição condicional de X dado Y $= j$ é idêntica à distribuição marginal de X (qualquer que seja y possível). De fato, temos:

$$P(X = i | Y = j) = \dfrac{P(X = i \text{ e }}{P(Y = j} = P(X = i)$$
\end{corollary}

\subsection*{Função de Probabilidade Acumulada}
\subsubsection*{Definição}
\begin{definition}
A \textbf{função de distribuição} (ou \textbf{função de probabilidade acumulada}) de X é definida por

$$F_X(x) = P(X \leq x)$$
\end{definition}

\begin{corollary}
Se F é a função de distribuição acumulada de uma variável aleatória discreta:

\begin{enumerate}[i.]
    \item F é não-decrescente;
    \item $F(- \infty) = 0$ e $F(+ \infty) = 1$ (ou seja, $\lim_{x\rightarrow{} - \infty} F(x) = 0$ e $\lim_{x\rightarrow{} + \infty} F(x) = 1$;
    \item F é constante por partes (isto é, uma função-escada).
\end{enumerate}
\end{corollary}

\begin{corollary}
Se F é uma função de distribuição acumulada de uma variável aleatória, então

$$P(a < X \leq b) = F(b) - F(a)$$
\end{corollary}

\subsubsection*{Quantis}
\begin{definition}
O q-quantil de uma variável aleatória X é qualquer valor $x_q$ onde a função acumulada "acerta" por q ou "passa" por p. Formalmente:

$$F(x_q -) \leq q \leq F(x_q)$$
\end{definition}

\subsection*{2.3 Valor Esperança}
\subsubsection*{Intuição e Definição}
\begin{definition}
Se X é uma variável aleatória discreta, definimos o \textbf{valor esperado} (ou esperança matemática, ou expectativa, ou média, ou valor médio) de X por

$$E(X) = \sum_{x \in S} x \cdot p(x)$$

isto é, $E(X)$ é uma média ponderada dos valores de X, com pesos iguais às respectivas probabilidades destes valores. Ocasionalmente, escrevemos $\mu_X = E(X)$

A esperança é uma medida de posição ou de tendência central (valores grandes de X acarretam $E(X)$ grande; valores pequenos de X acarretam $E(X)$ pequeno).
\end{definition}

\subsubsection*{Propriedades (Caso Unidimensional)}
\begin{corollary}
Se $Y = f(X)$, temos

$$E(Y) = E(f(X)) = \sum_{x \in S} f(x) \cdot p(x)$$
\end{corollary}

\begin{corollary}
Sejam a e b constantes quaisquer. Então:

$$E(a X + b) = a E(X) + b$$
\end{corollary}

\subsubsection*{Propriedades (Caso Bidimensional)}
\begin{corollary}
Se $Z = f(X, Y)$ então

$$E(Z) = E(f(X, Y)) = \sum_{x, y} f(x, y) p_{X, Y} (x, y)$$
\end{corollary}

\begin{corollary}
Sejam a, b e c constantes quaisquer. Então

$$E(a X + b Y + c) = a E(X) + b E(Y) + c$$
\end{corollary}

\begin{corollary}
Se X e Y são independentes, então $E(X Y) = E(X) E(Y)$.
\end{corollary}

\subsection*{2.4 Variância e Outras Medidas de Dispersão}
\subsubsection*{Definição}

\begin{definition}
    Duas medidas de dispersão comuns são o \textbf{desvio médio}, definidos por
    
    $$D M (X) = E(|X - E(X)|)$$
    
    e a \textbf{variância}, definida por
    
    $$Var(X) = E[(X - E(X))^2]$$
    
    Ao invés da variância, podemos medir a dispersão de X pelo seu \textbf{desvio-padrão}
    
    $$\sigma (X) = \sqrt{Var(X)}$$
\end{definition}

\begin{corollary}
    Sejam a e b constantes quaisquer. então
    
    $$Var(a X + b) = a^2 Var(X)$$
    $$\sigma (a X + b) = |a| \dot \sigma(X)$$
    $$D M(a X + b) = |a| \dot D M(X)$$
\end{corollary}

\begin{corollary}
    $$Var(X) = E(X^2) - (E(X))^2$$
\end{corollary}

\begin{corollary}
    Se X e Y são independentes, $Var(X + Y) = Var(X) + Var(Y)$
\end{corollary}

\subsubsection*{Desigualdade de Chebyshev}
\begin{theorem}[Desigualdade de Chebyshev]
    Seja X uma variavel aleatória com valor esperado $\mu = E(X)$ e desvio-padrão $\sigma = \sigma(X)$. Seja $P = \{ x \in \mathbb{R} | \ |x-\mu| < k \sigma \}$ (isto é, P é o intervalo aberto $(x - k \sigma, x + k \sigma)$, um conjunto de vlores de x que estão "perto da média" pelo menos k desvios-padrão). Então, para qualquer $k > 0$, tem-se
    
    $$P(X \notin P) \leq \dfrac{1}{k^2}$$
    
    ou seja
    
    $$P(|X - \mu| \geq k \sigma) \leq \dfrac{1}{k^2}$$
    $$P(|X - \mu| < k \sigma) \geq 1 - \dfrac{1}{k^2}$$
\end{theorem}

\subsection*{2.5 Covariância e Correlação}
\begin{definition}
    A \textbf{covariância} entre duas variáveis X e Y é 
    
    $$Cov (X, Y) = E[(X - E(X)) \dot (Y - E(Y))]$$
\end{definition}

\begin{corollary}
    $$Cov(X, Y) = E(XY) - E(X) E(Y)$$
\end{corollary}

\begin{corollary}
    Se X e Y são independentes, então $Cov(X, Y) = E(XY) - E(X) E(Y) = 0$
\end{corollary}

\begin{definition}
    Outra medida de "variação conjunta" de duas variáveis X e Y é a \textbf{correlação}
    
    $$\rho (X, Y) = \dfrac{Cov (X, Y)}{\sigma (X) \sigma(Y)}$$
\end{definition}

\subsubsection*{Um pouco de Álgebra Linear}
\begin{corollary}
    Para quaisquer variaveis aleatorias X e Y:
    
    $$Var(X + Y) = Var(X) + Var(Y) + 2 Cov (X, Y)$$
    
    E se X e Y são independentes vale que $Var(X + Y) = Var(X) + Var(Y)$.
\end{corollary}

\begin{corollary}
    $$Cov (aX + b, Y) = Cov (X, Y)$$
\end{corollary}

\begin{corollary}
    $$\rho (a X + b, Y) = \rho (X, Y) \text{ se } a > 0$$
    
    $$\rho (a X + b, Y) = - \rho (X, Y) \text{ se } a < 0$$
\end{corollary}

\section{Principais Distribuições Discretas}
\subsection{Distribuição}
Se os valores assumidos por uma certa variável aleatória X são equiprováveis dizemos que X tem um \textbf{distribuição uniforme}.

\subsection{Brevíssima Revisão de Analise Combinatória}
Aqui trata-se de conteúdo do ensino médio em caso de duvida \href{https://pt.wikipedia.org/wiki/Combinat\%C3\%B3ria}{see}

\subsection{Processo de Bernoulli}

\begin{definition}
    Um \textbf{processo de Bernoulli} é uma sequencia de experimentos com as seguintes características:
    
    \begin{enumerate}
        \item Cada experimento tem apenas dois resultados possíveis, denominados \textbf{sucesso} e \textbf{falha}
        
        \item Cada experimento tem a mesma probabilidade p de sucesso, e cada experimento é completamente independente de todos os outros.
    \end{enumerate}
\end{definition}

\subsubsection*{Distribuição Binomial}

\begin{definition}
    Suponha que o número de experimentos a serem feitos é determinado digamos, n experimentos. Seja X a variável aleatória que representa o numero de sucessos obtidos nestes n experimentos. Dizemos que X tem uma \textbf{distribuição binomial de parâmetros n e p} (e escrevemos X \backsim $Bin(n, p)$). Nesse caso a função de probabilidade de X passa a se chamar: BinomialDen
    
    $$P(X = k) = BinomialDen (k; n, p)$$
    
    e chamaremos a função acumulada de BinomialDist:
    
    $$P(X \leq k) = BinomialDist (k; n, p)$$
\end{definition}

\begin{corollary}
    $$BinomialDen (k;n, p) = P(X = k) = {n \choose k} p^k q^{n-k}$$
    
    $$BinomialDist (k; n, p) = P(X \leq k) = \sum_{i = 0}^k {n \choose k} p^k q^{n - k}$$
\end{corollary}

\begin{corollary}
    Seja $X \backsim Bin (n, p)$. Então
    
    $$E(X) = n p$$
    $$Var (X) = n p q$$
\end{corollary}

\subsubsection*{Distribuição Geométrica}
\begin{definition}
    Suponha que realizamos um processo de Bernoulli com probabilidade de sucesso de cada prova $p > 0$. Seja X o número de tentativas feitas até o primeiro sucesso (inclusive). Dizemos que X tem uma \textbf{distribuição geométrica de parâmetro} p, isto é, $X \backsim Geom (p)$
\end{definition}

\begin{corollary}
    Se $X \backsim Geom (p)$, então
    
    $$P(X = k) = Geom (k; p) = q^{k-1} p$$
    $$P(X \leq k) = 1 - q^k$$
\end{corollary}

\begin{corollary}
    Se $X \backsim Geom (p)$, então
    
    $$E(X) = \frac{1}{p}$$
    $$Var (X) = \frac{q}{p^2}$$
\end{corollary}

\subsubsection*{Distribuição Binomial Negativa}
\begin{definition}
    Suponha que o processo de Bernoulli é repetido até obter r sucessos. Seja X o número de tentativas feitas (incluindo o último sucesso). Dizemos que X tem uma \textbf{distribuição binomial negativa de parâmetros r e p}, isto é, $X \backsim NegBin(r, p)$. Note que, a distribuição geométrica é um caso particular dessa.
\end{definition}

\begin{corollary}
    Se $X \backsim NegBin(r, p)$, então para $k \geq r$ (k inteiro).
    
    $$P(X = k) = p. BinomialDen(r-1; k-1, p) = {k-1 \choose r-1} p^k q^{k - r}$$
    
    $$E(X) = \dfrac{r}{p}$$
    
    $$Var(X) = \dfrac{r q}{p^2}$$
\end{corollary}

\subsection*{3.4 Processo de Poisson}
\begin{definition}
    A \textbf{distribuição de Poisson} dizemos $X \backsim Poi (\mu)$ que no geral é $\lim_{n \rightarrow{} \infty} Bin (k; n, \frac{\mu}{n} = Poi (k; \mu)$
\end{definition}

\begin{corollary}
    Se $X \backsim Poi (\mu)$, tem-se
    
    $$P(X = k) = \frac{\mu^}{k!} e^{-\mu}$$
    
    $$E(X) = Var(X) = \mu$$
\end{corollary}

\subsection*{3.5 Distribuição Hipergeométrica}
\begin{definition}
    De uma caixa com r bolas "sucesso" e $N-r$ bolas "falha", extraímos sem reposição n bolas. Seja X o número de bolas sucesso. Dizemos que X tem \textbf{distribuição hipergeométrica com parâmetros n, r e N}, isto é, $X \backsim Hip (n, r, N)$.
\end{definition}
    
\begin{corollary}
    Se $X \backsim Hip (n, r, N)$, então
    
    $$P(X = k) = Hip (k; n, r, N) = \dfrac{{r \choose k}{N-r \choose n-k}}{{N \choose n}}$$
    
    $$E(X) = n p$$
    
    $$Var(X) = n p q \frac{N - m}{N -1}$$
    
    onde $p = \dfrac{r}{N}$
\end{corollary}

\section{Variáveis Aleatórias Continuas}
\subsection*{4.1 Distribuições Contínuas}
\subsubsection*{Função de Distribuição Acumulada}

\begin{definition}
    A \textbf{função de distribuição acumulada (função de distribuição; fda)} de uma variável aleatória X é
    
    $$F_X(x) = P(X \leq x)$$
\end{definition}

\begin{corollary}
    Se $F(x)$ é a f.d.a. de uma variavel read X, então
    
    $$F é \text{não-decrescente}$$
    
    $$F(- \infty) = P(X \leq - \infty) = 0 \ \text{ e } \ F(+\infty) = P(X \in \mathbb{R)} = 1$$
    
    $$P(a \leq X \leq b) = F(b) - F(a)$$
    
    Note que, no caso continuo $P(a < X \leq b) = P(a \leq X \leq b)$, pois $P(X = a) = 0$.
\end{corollary}

\subsubsection*{Quantis}

\begin{definition}
    O q-quantil de uma variável aleatória X é qualquer valor $x_q$ onde a função acumulada "acerta" q. Formalmente $F(x_q) = q$
\end{definition}

\subsubsection*{Função Densidade de Probabilidade}

\begin{definition}
    A \textbf{função densidade de probabilidade (fdp)} de X é a derivada da função acumulada:
    
    $$f_X(x) = \lim_{\Delta x \rightarrow{} 0} \dfrac{F(x + \Delta x) - F(x)}{\Delta x} = F'_X (x)$$
\end{definition}

\begin{corollary}
    Dada a f.d.p. de uma variável aleatória contínua X, encontramos probabilidades pela fórmula
    
    $$P(a \leq X \leq) = F(b) - F(a) = \int_a^b f(t) d t$$
    
    Em particular, como $F(- \infty) = 0$, note que
    
    $$F(x) = P(X \leq x) = \lim{-\infty}^x f(t) d t$$
\end{corollary}

\begin{corollary}
    Se f(x) é a f.d.p. de uma variável aleatória real X, então para todo x real:
    
    $$0 \leq f(x)$$
    
    $$\int_{- \infty}^{\infty} f(t) d t = 1$$
\end{corollary}

\begin{definition}
    A \textbf{moda} de uma variável aleatória é o valor x onde a densidade f(x) é máxima.
\end{definition}

\subsubsection*{Funções de Variáveis Aleatórias Contínuas}

Seja X uma variável aleatória de densidade f(x) e seja $Y = h(X)$ onde h é uma função crescente. Então a densidade g(y) da variável Y satisfaz

$$g(y) = \dfrac{f(x)}{h'(x)} = \dfrac{f(x)}{\frac{d y}{d x}}$$

onde $x = h^{-1} (y)$, ou seja,

$$g(y) d y = f(x) d x$$

\subsection*{Valor Esperado e Variância}
\subsubsection*{Valor Esperado}

\begin{definition}
    Se X é uma variável com densidade f(x), definimos seu \textbf{valor esperado (valor médio, esperança)}, por
    
    $$\mu = E(X) = \int_{- \infty}^{\infty} x f(x) d x$$
\end{definition}

\begin{corollary}
    Se $Y = h(X)$, então 
    
    $$E(Y) = E(h(X)) = \int_{- \infty}^{\infty} h(x) f(x) d x$$
\end{corollary}

\begin{corollary}
    Sejam a e b constantes quaisquer. Então
    
    $$E(a X + b) = a E(X) + b$$
\end{corollary}

\subsubsection*{Variância}

\begin{definition}
    A \textbf{variância} e o \textbf{desvio-padrão} de uma variável aleatória X com densidade f(x) e média $E(X) = \mu$ são
    
    $$Var(X) = E \left ((X - \mu)^2 \right ) = \int_{- \infty}^{\infty} (x - \mu)^2 f(x) d x$$
    
    $$\sigma(X) = \sqrt{Var (X)}$$
\end{definition}

\begin{corollary}
    $$Var(a X + b) = a^2 Var(X)$$
    
    $$Var(X) = E(X^2) - (E(X))^2$$
\end{corollary}

\begin{theorem}[Desigualdade de Chebyshev]
    Esse teorema está definido em 2.4, de forma igual.
\end{theorem}

\end{document}
